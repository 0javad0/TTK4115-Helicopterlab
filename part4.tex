%%% Local Variables:
%%% mode: latex
%%% TeX-master: "report_main"
%%% End:

\section{Part 4 -- State estimation}
This section consists of the development of an observer to estimate the
nonmeasured angular velocities.
%
\subsection{Problem 1}
By describing the system in \cref{eq:linearized EoM} in the following
state-space form
%
\begin{align}
  \begin{split}
    \dot{\bm{x}} &= \bm{Ax} + \bm{Bu} \\
    \bm{y} &= \bm{Cx}
  \end{split}
\end{align}
%
where  $\bm{A}$, $\bm{B}$ and $\bm{C}$ are matrices. The state -,
input - and output vector are given by
%
\begin{equation}
  \label{eq:state_space_vectors}
  \bm{x} =
  \begin{bmatrix}
    \tilde{p} \\
    \dot{\tilde{p}} \\
    \tilde{e} \\
    \dot{\tilde{e}} \\
    \tilde{\lambda} \\
    \dot{\tilde{\lambda}} \\
  \end{bmatrix}
  , \quad \bm{u} =
  \begin{bmatrix}
    \tilde{V_s} \\
    \tilde{V_d} \\
  \end{bmatrix}
  \quad \text{and} \quad \bm{y} =
  \begin{bmatrix}
    \tilde{p} \\
    \tilde{e} \\
    \tilde{\lambda}\\
  \end{bmatrix}
\end{equation}
%
This gives the following $\bm{A}$, $\bm{B}$ and $\bm{C}$ matrices
%
\begin{equation}
  \label{eq:state_space_A_B_C}
  \bm{A} =
  \begin{bmatrix}
    0 & 1 & 0 & 0 & 0 & 0 \\
    0 & 0 & 0 & 0 & 0 & 0 \\
    0 & 0 & 0 & 1 & 0 & 0 \\
    0 & 0 & 0 & 0 & 0 & 0 \\
    0 & 0 & 0 & 0 & 0 & 1 \\
    K_3 & 0 & 0 & 0 & 0 & 0 \\
  \end{bmatrix}
  , \quad \bm{B} =
  \begin{bmatrix}
    0 & 0 \\
    0 & K_1 \\
    0 & 0 \\
    K_2 & 0 \\
    0 & 0 \\
    0 & 0 \\
  \end{bmatrix}
  \quad \text{and} \quad \bm{C} =
  \begin{bmatrix}
    1 & 0 & 0 & 0 & 0 & 0 \\
    0 & 0 & 1 & 0 & 0 & 0 \\
    0 & 0 & 0 & 0 & 1 & 0 \\
  \end{bmatrix}
\end{equation}
%
Where $K_1$, $K_2$ and $K_3$ are given by \cref{eq:linearized EoM}.
%
\subsection{Problem 2}

\subsection{Problem 3}
The observer matrix can be used. Because $\bm{C}$ is 3x6, the observer matrix becomes:
\begin{align*}
	\bm{\mathcal{O}} = 
	\begin{bmatrix}
		\bm{C} \\
		\bm{CA} \\
	\end{bmatrix},
\end{align*}
which is 6x6. For only $\tilde{e}$ and $\tilde{\lambda}$ measured, the observer matrix becomes:
\begin{equation}
	\bm{\mathcal{O}} =
  \begin{bmatrix}
    1 & 0 & 0 & 0 & 0 & 0 \\
    0 & 0 & 0 & 0 & 0 & 0 \\
    0 & 0 & 0 & 0 & 1 & 0 \\
		etc
	\end{bmatrix}
\end{equation}
		