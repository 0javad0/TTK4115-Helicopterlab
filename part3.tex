\section{Part 3 - Multivariable control}
\subsection{Problem 1}
The matrices A and B are:
\begin{equation}
  \boldsymbol{A} = \begin{bmatrix}
    0 & 1 & 0 \\
    0 & 0 & 0 \\
    0 & 0 & 0 \\
  \end{bmatrix}, \hspace{0.5cm}
  \boldsymbol{B} = \begin{bmatrix}
    0 & 0 \\
    0 & K_1 \\
    K_2 & 0 \\
  \end{bmatrix}
\end{equation}

\subsection{Problem 2}
The systems controllability is examined through its controllability
matrix $\boldsymbol{\mathcal{C}}$:

\begin{equation}
  \boldsymbol{\mathcal{C}} = \begin{bmatrix}
    \boldsymbol{B} & \boldsymbol{AB}
  \end{bmatrix}
  =
  \begin{bmatrix}
    0 & 0 & 0 & K_1 \\
    0 & K_1 & 0 & 0 \\
    K_2 & 0 & 0 & 0 \\
  \end{bmatrix}
\end{equation}
which has full rank: $rank(\boldsymbol{\mathcal{C}}) = 3$, and is thus
controllable.

Here, we are using LQR with reference feed-forward in order to control
our system. Our $\boldsymbol{P}$ is defined such that as time goes to
infinity, our states tend to the reference values. This happens when
$\dot{\boldsymbol{x}} = 0$, as the system reaches a stable equilibrium
around the reference values:

\begin{align*}
  \dot{\boldsymbol{x}} &= \boldsymbol{Ax} - \boldsymbol{Bu} \\
                       &= \boldsymbol{Ax} -
                         \boldsymbol{B}(\boldsymbol{Pr} -
                         \boldsymbol{Kx}) = 0
\end{align*}
Which implies:
\begin{align*}
  \dot{\boldsymbol{x}} &= \boldsymbol{Ax} - \boldsymbol{Bu} \\
                       &= \boldsymbol{Ax} -
                         \boldsymbol{B}(\boldsymbol{Pr} -
                         \boldsymbol{Kx}) = 0
\end{align*}
\subsection{Problem 3}
\subsection{Problem 4}

%%% Local Variables:
%%% mode: latex
%%% TeX-master: "report_main"
%%% End:
