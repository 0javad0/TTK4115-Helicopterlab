%%% Local Variables:
%%% mode: latex
%%% TeX-master: "report_main"
%%% End:

\section{Part 2 -- Monovariable control}
\subsection{Problem 1}
We are given the controller shown in \cref{eq:pd_controller}.
\begin{equation}
  \label{eq:pd_controller}
  \tilde{V_d} = K_{pp}(\tilde{p_c} - \tilde{p}) - K_{pd} \dot{\tilde{p}}
\end {equation}
We take this controller and substitute it in the equation for pitch
angle (\cref{eq:pitch}).
\begin{equation}
  \label{eq:pitch_with_pd}
  \ddot{\tilde{p}} = K_1 K_{pp}(\tilde{p_c} - \tilde{p}) - K_1 K_{pd}
  \dot{\tilde{p}}
\end{equation}
Now we Laplace transform \cref{eq:pitch_with_pd} to find the transfer
function $\frac{\tilde{p}(s)}{\tilde{p_c}(s)}$.
\begin{align*}
  \ddot{\tilde{p}} + K_1 K_{pd}\dot{\tilde{p}}
  + K_1K_{pp}\tilde{p} &= K_1 K_{pp}\tilde{p_c} \\
  \mathcal{L}\rightarrow&  \\
  s^2\tilde{p}(s) + sK_1K_{pd}\tilde{p}(s)
  + K_1K_{pp}\tilde{p}(s) &= K_1K_{pp}\tilde{p_c}(s)
\end{align*}
Which gives us our transfer function
\begin{equation}
  \label{eq:trans_func}
  \frac{\tilde{p}(s)}{\tilde{p_c}(s)} = \frac{K_1K_{pp}}{s^2+K_1K_{pd}s+K_1K_{pp}}
\end{equation}

The linearized pitch dynamics can be regarded as a second-order linear
system, which means that if we place \cref{eq:trans_func} on the form
shown in \cref{eq:sol_system} we can determine $K_{pp}$ and $K_{pd}$
from $\omega$ and $\zeta$.
\begin{equation}
  \label{eq:sol_system}
  h(s) = \frac{\omega^2}{s^2+2\zeta\omega^2s+\omega^2}
\end{equation}

This gives us the following relations
\begin{align}
  \label{eq:omega}
  \omega &= \sqrt{K_ 1K_ {pp}} \\
  2\zeta\omega^2 &= K_ 1K_ {pd} \nonumber \\
  \label{eq:zeta}
  \zeta = \frac{K_ 1K_ {pd}}{2\omega^2} &= \frac{K_{pd}}{2K_{pp}}
\end{align}

We know that for a critically damped system $\zeta = 1$, which gives
us

\begin{equation}
  \label{eq:K_pd}
  K_{pd} = 2K_{pp}
\end{equation}

We chose a $K_{pp} = 3$ and then from the relation in \cref{eq:K_pd}
we get $K_{pd} = 6$. With these values the response of the pitch angle
to the input was slower than desired. Therefore, $K_{pp}$ was
increased to $K_{pp} = 12.5$ and $K_{pd}$ was lowered to underdamp the
system, until it  was sufficiently responsive at $K_{pd} = 0.7K_{pp} =
8.75$. At these values the system responded faster with only minor
oscillations. It was observed that larger values of $K_{pp}$ gave rise
to larger oscillations.

\begin{figure}[H]
  \caption{Change in pole position by increasing $K_{pp}$ given a constant $K_{pd}$}
  \label{fig:root_locus}
  \includegraphics[width=\textwidth]{images/root_locus}
\end{figure}

At the critically damped point, the poles lie on the same point on
the x-axis. When $K_{pp}$ is decreased in relation to $K_{pd}$ the system is
over-damped the poles move away from each other along the x-axis.
When $K_{pp}$ is increased in relation to $K_{pd}$, the system is under damped
and the poles move away from each other vertically from the critically
damped point.

With the PD controller, it was significantly easier to control the
helicopter than with just feed forward joystick control.


\subsection{Problem 2}
By plugging the P controller for travel: 
\begin{equation}
	\tilde{p} = K_{rp}(\dot{\tilde{\lambda}}_c - \dot{\tilde{\lambda}})
\end{equation}
into the equation of motion for travel \todo{Add equation reference}, the transfer function between $\dot{\tilde{\lambda}}$ and $\dot{\tilde{\lambda}}_c$  can be derived. By substituting the right side of the controller into the equation of motion for travel the equation becomes:
\begin{equation}
	\ddot{\tilde{\lambda}} = K_3(K_{rp}(\dot{\tilde{\lambda}}_c - \dot{\tilde{\lambda}}))
\end{equation}
After taking the laplace transform and rearranging terms, the transfer function is as follows:
\begin{equation}
	\frac{\ddot{\tilde{\lambda}}(s)}{\dot{\tilde{\lambda}}_c(s)} = \frac{K_3K_{rp}}{s + K_3K_{rp}}
\end{equation}
This controller was quite quick and stable after correctly tuning $K_{rp}$. The value that was deemed best was $K_{rp}$ = \todo{fill in this value}. Higher values of $K_{rp}$ led to \todo{Describe performance}, while lower values of $K_{rp}$ led to \todo{describe performance}.  It was not nessecary to add a gain to the x value of the joystick to produce quick and accurate control.


