%%% Local Variables:
%%% mode: latex
%%% TeX-master: "report_main"
%%% End:

\def\SPSB#1#2{\rlap{\textsuperscript{{#1}}}\SB{#2}}
\def\SP#1{\textsuperscript{{#1}}}
\def\SB#1{\textsubscript{{#1}}}


\section{Part 1 - Mathematical modeling}
\subsection{Problem 1}

We use the helicopter model \cref{fig:helicopter_model} as our starting point for deriving the equations of motion.
\begin{figure}[hbp]
\caption{the helicopter model figure 7 from the assignment \cite[p.12]{assignment} with relevant distances drawn in.}
\label{fig:helicopter_model}
\includegraphics[width=\textwidth]{images/helicopter_model}
\end{figure}

The equations of motion for the pitch is the momentum around the point p in the clockwise direction as shown in \cref{fig:helicopter_model}:
\begin{align*}
J_p\ddot{p} &= l_p(F_{g,b} - F_b - F_{g,f} + F_f) \\
						&= l_p(m_pg - mp_g + K_fV_f - V_b) \\
						&= l_pK_f(V_f-V_b)
\end{align*}
Since $V_d = V_f-V_b$, we can write this as:
\begin{equation}
J_p\ddot{p} = l_pK_fVd
\end{equation}
Here, we can see that $L_1 = l_pK_f$.

\begin{figure}[H]
\caption{the elevation model}
\label{fig:elevation_model}
\includegraphics[width=0.5\textwidth]{images/elevation_model}
\end{figure}

\subsection{Problem 2}
	To linearize the system about the point with all state variables equal to zero ($p, e, \lambda$)\SP{T} = (\textit{$\dot{p}$, $\dot{e}$,$\dot{\lambda}$})\SP{T} = (0, 0, 0))  the inputs in the equations of motion (V\SPSB{*}{s} and V\SPSB{*}{d}) must be set to values that make this an equilibrium point.
	At the linearization point, the equation of motion for pitch (\todo{ADD EQUATION REFERENCE}) reduces to $0 = L\SB{1} * V\SPSB{*}{d}$ therefore:
	\begin{equation}
		V\SPSB{*}{d} = 0
	\end{equation}
	At the linearization point, the equation of motion for elevation (\todo{ADD EQUATION REFERENCE}) reduces to $0 = L\SB{2} + L\SB{3}*V\SPSB{*}{s}$ therefore:
	\begin{equation}
	V\SPSB{*}{s} = -L\SB{2} / L\SB{3}
	\end{equation}
While the equation of motion for travel (\todo{ADD EQUATION REFERENCE}) at the linearization point simply reduces to 0 = 0.
	\\The following transformation is performed to simplify the analysis \todo{Add cite of assignment here}:
	\begin{equation}
		\begin{bmatrix}
			\tilde{p} \\
			\tilde{e} \\
			\tilde{\lambda}
		\end{bmatrix} 
		=
		\begin{bmatrix}
			p \\
			e \\
			\lambda
		\end{bmatrix} 
		- 
		\begin{bmatrix}
			p\SP{*} \\
			e\SP{*} \\
			\lambda\SP{*}
		\end{bmatrix}
		and
		\begin{bmatrix}
			\tilde{V}\SB{s} \\
			\tilde{V}\SB{d}
		\end{bmatrix}
		=
		\begin{bmatrix}
			V\SB{s} \\
			V\SB{d}
		\end{bmatrix} 
		- 
		\begin{bmatrix}
			V\SPSB{*}{s} \\
			V\SPSB{*}{d} 
		\end{bmatrix}
	\end{equation}
	The equations of motion in the transformed system are therefore:
	\begin{subequations}
	\begin{align}
		J_p\ddot{\tilde{p}} &= L_1\tilde{V_d} \\
		J_e\ddot{\tilde{e}} &= L_2cos(\tilde{e}) + L_3(\tilde{V_s}+L_2/L_3)cos(\tilde{p}) \\
		J_{\lambda}\ddot{\tilde{\lambda}} &= L_4(\tilde{V_s}+L_2/L_3)cos(\tilde{e})sin(\tilde{p})
	\end{align}
	\end{subequations}
	By choosing the state to be x = ($\tilde{p}$, $\tilde{e}$, $\tilde{\lambda}$, \textit{$\dot{p}$, $\dot{e}$,$\dot{\lambda}$}) the nonlinear state equations become:
	\begin{subequations}
	\begin{align}
		\dot{x}_1 &= x_4 \\
		\dot{x}_2 &= x_5 \\
		\dot{x}_3 &= x_6 \\
		\dot{x}_4 &= (L_1/J_p) V_d \\
		\dot{x}_5 &= (L_2/J_e)cos(x_2) + (L_3/J_e)(V_s + L_2 / L_3)cos(x_1) \\
		\dot{x}_6 &= (L_4 / J_\lambda) (V_s + L_2 / L_3)cos(x_2)sin(x_1)
	\end{align}
	\end{subequations}
	If the above system is expressed as $\dot{x} = h(x, u)$, where x is the state and u is the input, the system is linearized by finding the Jacobians of h with respect to the state and the input and then plugging in the equilibrium values.
	\begin{equation}
		\frac{\partial h}{\partial x} = A =
		\begin{bmatrix}
			0 & 0 & 0 & 1 & 0 & 0 \\
			0 & 0 & 0 & 0 & 1 & 0 \\
			0 & 0 & 0 & 0 & 0 & 1 \\
			0 & 0 & 0 & 0 & 0 & 0 \\
			0 & 0 & 0 & 0 & 0 & 0 \\
			\frac{L_4L_2}{J_\lambda L_3} & 0 & 0 & 0 & 0 & 0 
		\end{bmatrix}
		\qquad
		\frac{\partial h}{\partial u} = B = 
		\begin{bmatrix}
			0 & 0 \\
			0 & 0 \\
			0 & 0 \\
			0 & L_1/J_p \\
			L_3/J_e & 0 \\
			0 & 0  
		\end{bmatrix}
	\end{equation}
where $L_1, L_2, L_3$ and $L_4$ were calculated in Section 1.1 \todo{maybe cite this} and $J_p, J_e$ and $J_\lambda$ were given in the assignment description \todo{add cite to assignment}.
The linearized equations of motion can therefore be written in the following form:
\begin{subequations}
\begin{align}
\ddot{\tilde{p}} &= K_1\tilde{V}_d \qquad K_1 = \frac{L_1}{J_p}\\
\ddot{\tilde{e}} &= K_2\tilde{V}_s \qquad K_2 = \frac{L_3}{J_e}\\
\ddot{\tilde{\lambda}} &= K_3\tilde{p} \qquad K_2 = \frac{L_4L_2}{J_\lambda L_3} 
\end{align}
\end{subequations}
\subsection{Problem 3}
\subsection{Problem 4}

	